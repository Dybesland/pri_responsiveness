% All neds to be changed: 
% Figure out which models to include and include controlls (Party/Government) 
% Present graphs and tables in a scientific standard
% Update all estimates
% Check text. 

\section*{Results}
\subsection*{Gender Differences in Citizens' Issue Priorities}
Since gender differences at the citizen level are a logical prerequisite for gender-biased priority responsiveness, the first step was to test whether men and women have different issue priorities. Figure x  shows the estimated odds ratios for women to prioritise an issue compared to men. I obtained the estimates running separate models for each policy issue, controlling for age, level of education, and county of residence. All models include time-fixed effects, survey weights, as well as random intercepts at the individual level.

Figure z  reveals that women are more likely to prioritise issues such as family, health, equality, education, the elderly, and environmental policies. Conversely, they are significantly less likely to prioritise labour, immigration, regional politics, foreign politics, freedom, trade, the economy, and infrastructure when compared to men. These results support $H1$ positing that men and women in the Norwegian public display significantly different issue priorities. Furthermore, the analyses reveal that gender differences mostly corroborate with theories about gendered interests. They also seem to be related to aspects of Norwegian society with more pronounced gender roles. Moreover, the figure shows that there is a greater variety in the issues most likely to be prioritised by men. Among the issues with significant gender differences, women are more likely to prioritise six of them, whereas men exhibit a higher likelihood to prioritise twelve issues.\footnote{\color{red} The complete model results will be presented in the Appendix.}

The odds ratios compare the likelihood between genders, providing a relative measure of how much more likely one gender is to prioritise an issue compared to the other. Hence they offer insights into the relative importance between genders but do not provide information on the absolute importance of the issue to each gender. To understand the absolute level of importance attributed to various issues among men and women, we can look at the shares as presented in . 


In Figure z , shares represent the proportion of each gender prioritising specific issues in each survey round, thereby directly measuring priority levels within each gender group. Observing the trends, we see that the gender differences in issue priorities are quite stable over time, further supporting $H1$ and corroborating the results from the logistic regressions.

also testifies to why it is important to also provide a rank measure for citizens priorities. A rank measure gives information about the relative importance between issues for each gender, but not the intensity of priorities which the shares measures. A difference in ranking between genders for the same issues give indications for areas where men and women diverge significantly in their prioritisation, even though the share of individuals with in that gender displays as increase in prioritising that issue from one point in time to another in the appendix for the rankings of each issue per gender over time). 

% All this needs to be changed 
% Figure out how to deal with party (fixed effects, subsets, covariate with time fixed effects)
% Macro/meso level variables can be included as controls after, such as covid, time fixed effects for responsiveness. 

% Perhaps fever measures of the same thing in the main text, and combined table with male/female representatives responsiveness. 

\subsection*{Gender Differences in Priority responsiveness}
Given the identified significant gender differences in issue priorities at the citizen level, the subsequent step involves empirically evaluating $H2_{a}$ and $H2_{b}$, concerning whether male and female citizens benefit from equal levels of priority responsiveness. This investigation seeks to ascertain whether a gender-based disparity exists in how political representatives address the issues deemed important by their constituents.

The results of the Fractional Logit models, employed to estimate the responsiveness of representatives, are depicted in. The analyses utilise the proportion of all speeches addressing specific issues between successive survey rounds as the dependent variable, thereby capturing representative attention in the periods following each citizen survey. Independent variables reflect citizen priorities, including total shares and gender-specific rankings, along with a calculated gender gap (men-women) in priorities. This approach assesses whether representatives exhibit higher responsiveness to topics as the discrepancy between male and female citizens' priorities increases


The analysis presented in reveals that representatives generally appear to be responsive to the collective priorities of citizens, as indicated by the significant coefficient of for the total share in Model 1. This suggests a strong relationship between the aggregated public concerns and the subsequent attention these concerns receive in parliamentary speeches, without disaggregation by gender.

However, a more nuanced picture emerges upon testing the effect or gender-specific priorities. Model 2 shows that the priorities of men, when measured as the aggregated share of men’s priorities of each issue from specific survey rounds, significantly correlate with the focus of representatives ($\beta = 5.57$, standard error = 0.77, $p < .01$). This is contrary to the priorities of women, which do not exhibit a similar level of impact on representatives priorities ($\beta = -0.69$, standard error = 0.61, $p > .05$). This indicates that, in contexts where men’s priorities are accounted for, they are stronger predictors of legislative attention in parliamentary speeches compared to women’s priorities.

Additionally, Model 3, which assesses the impact of the gender gap in shares (men-women), indicates no significant relationship between the gender gap and representatives’ issue priorities ($\beta = 0.40$, standard error = 0.78, $p > .05$). This suggests that merely the difference between men’s and women’s share of priority does not significantly predict parliamentary attention.

Turning to the analyses based on rankings rather than shares, the results continue to illustrate gendered patterns of responsiveness. Notably, Model 5 shows a significant influence of men’s rankings on parliamentary attention ($\beta = 0.05$, standard error = 0.01, $p < .01$), while women’s rankings do not present the same level of significance ($\beta = -0.01$, standard error = 0.01, $p > .05$).

Furthermore, Model 6 reveals a significant impact of the gender gap in rankings (men-women) on legislative attention ($\beta = 0.03$, standard error = 0.01, $p < .01$). This indicates that when issues are ranked higher by men compared to women, these issues are more likely to be addressed in parliament,thereby suggesting a gendered discrepancy in political responsiveness where the ranking priority difference between genders influences the likelihood of an issue being addressed. \cref{fig:fig4} illustrates how the probability of representatives attention to issues changes in response to gender gaps in priorities measured as shares and rankings.



\subsection*{Male and female representatives issue attention}
The analyses presented in \cref{table1} support $H2_{a}$ suggesting that male citizens benefit from higher levels of priority responsiveness than female citizens.
Turning now to the potential explanations for why female citizens do not appear to benefit from the same level of priority responsiveness as male citizens, I present results from the analyses conducted to test Hypothesis 3 ($H3$), which posits that representatives are more responsive to citizens of their own gender in the issues they address during parliamentary debates. The analyses shown in \cref{table2} and \cref{table3} were conducted in the same way as those investigating overall responsiveness but on subsets of speeches from female and male representatives, respectively.


\cref{table2}, Models 1 through 6, show the estimated relationships between citizens' priorities and female representatives' issue attention. As demonstrated in Model 1, the total share of citizens' priorities significantly impacts the issue priorities of female representatives, with a coefficient of 1.92 ($p < .01$), suggesting that female representatives are responsive to the collective public opinion.

Contrary to the expectations of $ H3 $, there is no clear indication that female citizens' issue priorities are a stronger predictor for female representatives' attention to specific issues than those of male citizens. Specifically, Model 2 shows a non-significant influence of women's share (coefficient = 0.70, p > .05) compared to men's share (coefficient = 1.32, p > .05), indicating no statistically significant gender-based preferences in responsiveness among female representatives. Furthermore, the gap share between men and women (Model 3, coefficient = -0.97, p > .05) does not significantly predict female representatives' issue priorities, which undermines the hypothesis that female representatives would favour issues more prioritised by women.

Similarly, when examining priorities measured as rankings (Models 4 to 6), we observe that neither women's nor men's rankings present a significantly stronger predictor for the issues female representatives choose to address. Model 4, showcasing the total rank, shows a minor but significant effect (coefficient = 0.02, $p < .01$), indicating some responsiveness to the overall ranking of issues, regardless of the gender.

For male representatives, however, the results appear differently. As shown in Table x , male representatives demonstrate stronger responsiveness to citizens' overall priorities compared to their female counterparts, as judged by the magnitudes of the coefficients from the first models in both and Table x . This difference in responsiveness is primarily driven by their attention to male citizens. Specifically, the negative coefficient for women's share ($\beta = -2.36$, standard error = 0.77, $p < .01$) suggests that male representatives are less likely to prioritise issues important to a larger share of female constituents, when the priorities of male citizens are accounted for. Conversely, male citizens' priorities are a strong predictor of male representatives' issue attention ($\beta = 6.45$, standard error = 0.95, $p < .01$).




The finding that male citizens' priorities are a strong predictor of male representatives' issue attention gains further support from Models 3 and 6, which demonstrate that the more significant the disparity in priorities between genders (men's priorities minus women's priorities), both when measured as shares and rankings, the more likely male representatives are to address concerns more pertinent to their male constituents. The coefficients in these models ($\beta = 2.50$ for the gap in shares and $\beta = 0.05$ for the gap in rankings, both with $p < .01$) underscore an apparent gender-based discrepancy in political responsiveness, with male representatives showing a significant tendency to align more closely with the priorities of male citizens. 

Figure x illustrates the variation in issue attention by male and female representatives relative to the priority gap between male and female citizens. It demonstrates that as the priority gap on a specific issue widens—indicating that the issue becomes more important to male citizens compared to female citizens—male representatives progressively address topics that align more closely with the aggregated priorities of male citizens rather than those of female citizens. From \cref{fig:fig5}, we observe that when both gender groups attribute the same level of priority to an issue (resulting in no priority gap), the probability of both male and female representatives to address this issue is approximately 5 per cent. However, as an issue gains importance among male citizens, responsiveness from male representatives correspondingly increases. Conversely, female representatives exhibit no significant tendency to adjust their attention disproportionately in response to the prioritisation by one gender over the other. 


In sum, the examination of parliamentary speeches indicates a higher responsiveness among politicians to the concerns predominantly raised by male citizens. However, female representatives do not show a significant inclination towards prioritizing the issues of women citizens. Instead, the analyses suggest that male representatives disproportionately respond to the priorities of male constituents. This pattern of male representatives' biased responsiveness appears to be a key factor contributing to the overall unequal attention Norwegian representatives give to their citizens' priorities.