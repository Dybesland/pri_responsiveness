% Comment from Slapin: 
% Sell this better from the start. This is new. 

% Slapin and Immergut: Justify gender. 

\section*{Introduction}
Two fundamental principles stand as the pillars of modern democracy — responsive government and political equality. These ideals emphasize the government's responsibility to ensure that policies reflect the wishes of its citizens, without any systematic bias favoring specific groups \parencite{dahl_polyarchy_1971}. Despite these principles, evidence from Western democracies suggests that certain groups are more likely than others to have their preferences translated into policy. Studies have demonstrated that policymakers' positions and policy decisions tend to align more closely with the desires of wealthier and more highly educated citizens, as opposed to those with lower socioeconomic status \parencite{gilens_affluence_2012, bartels_unequal_2008, schakel_real_2020, schakel_degrees_2021, mathisen_affluence_2023}.

Interestingly, research on gender biases in responsiveness is less conclusive. While several studies indicate that men's preferences are more likely reflected in policy outcomes than women's \parencite{persson_mans_2023,reher_gender_2018,mathisen_influence_2024,kopkin_gender_2023}, research focusing on earlier stages of the policy-making process shows smaller gender gaps \parencite{ferland_gender_2020,bernauer_mind_2015,dingler_parliaments_2019, homola_are_2019}, potentially hinting at evolving representational gaps through the policy cycle.

Studies on gender biases in responsiveness share a common focus; they primarily examine positional responsiveness, investigating whether the direction of a policy or legislators' positions reflect the directional preferences of citizens equally well \parencite{persson_mans_2023,reher_gender_2018,mathisen_influence_2024,kopkin_gender_2023,ferland_gender_2020, bernauer_mind_2015,homola_are_2019, dingler_parliaments_2019}. This paper focuses on a less explored aspect of biased responsiveness: priority responsiveness, referring to whether the political issue priorities of men and women are equally reflected in policymakers' issue priorities.\footnote{\textcite{bevan_representation_2014} refer to the reflection of citizens' in governments issue priorities as dynamic agenda responsiveness. The same concept is sometimes referred to as issue representation \parencite{dennison_explaining_2023, kluver_setting_2016}, or agenda responsiveness \parencite{traber_social_2022,alexandrova_agenda_2016}. I choose to use priority responsiveness \parencite{reher_gender_2018} as it fits best with other concepts of responsiveness and congruence discussed in this paper.}

The connection between public opinion and policy output is a complex, multi-staged process, and a prerequisite for policy change is that issues must first be introduced into the political system and receive attention \parencite{jones_representation_2004, jones_representation_2009}. Therefore, the government's responsiveness to citizens' issue priorities is a crucial aspect of political representation, functioning as a potential mechanism linking public opinion and policy. However, given politicians' limited time and capacity to address all issues, policy-making involves trade-offs where policymakers must prioritise certain issues over others \parencite{jones_agenda_2018} The emphasis placed on various issues during policy-making may be influenced by  institutional constraints and party control \parencite{slapin_introduction_2014}, incentives of vote maximisation, and party competition \parencite{hobolt_government_2008, abou-chadi_brahmin_2021}, as well as information (biases) about constituents' concerns \parencite{traber_social_2022}. Such trade offs may potentially result in biases toward the priorities of specific groups.

In this paper, I show that there are significant disparities in the importance that male and female citizens place on a broad range of policy issues and argue that equal priority responsiveness is a crucial aspect of democratic representation overall. If the issues considered important by one group are systematically prioritised over those deemed important by another, then representation is not truly equal. For as \textcite[2]{jones_representation_2004} argue, "How representative is a legislative action that matches the policy preferences of the public on a low priority issue but ignores high priority issues?" Thus, evaluating democratic representation based solely on positional responsiveness — whether the direction of a policy or politicians’ positions reflects the directional preferences of citizens — leaves a blind spot by overlooking the importance that citizens place on specific policy issues.

I examine gender-based differences in priority responsiveness by exploring whether representatives are equally likely to prioritise the issues important to male and female citizens during legislative speeches. To do so, I combine a decades worth of data on public priorities from the Norwegian Citizen Panel (NCP) \parencite{ivarsflaten_norwegian_2024} with a comprehensive time-series speech dataset comprising 111,477 parliamentary speeches delivered by representatives covering the same period (2013-2023). I Employ computational text analysis within the framework of semi-supervised topic modeling, to assess the issue priorities of representatives. By integrating estimates of citizens' and representatives' issue priorities at various time points, I test whether one gender benefits from higher levels of priority responsiveness than the other.

Furthermore, I explore whether political representatives are more likely to address the issue priorities of citizens with whom they share the same gender. In doing so, I build on a vast body of research that examines the conditions under which women experience substantive representation, typically understood as representatives acting "in the interest of the represented" \parencite{pitkin_concept_1967,mansbridge_should_1999}. Thereby considering whether there is a relationship between women's descriptive and substantive representation, implying that the presence of women in parliament leads to better representation of women's interests \parencite{clayton_quotas_2017, bolzendahl_womens_2007, wangnerud_women_2009}. 

The findings of this paper can be summarised as follows: the analyses of parliamentary speeches suggest that politicians are more responsive to the issue priorities of male citizens. Interestingly, there are no indications that female representatives address women citizens’ priorities to a greater extent. However, male representatives cater disproportionately to the priorities of their male constituents. Male representatives' unequal responsiveness to citizens' issue priorities is what appears to drive the unequal responsiveness of Norwegian representatives towards citizens' political issue priorities. 