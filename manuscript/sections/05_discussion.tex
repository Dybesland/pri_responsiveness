% Should be rewritten
% Include a paragraph on the concentration of issues among female citizens, and how the parliament might adress a wider range of issues, and therfore cater better to men. 

\section*{Conclusion}
Despite the importance of having the issues that one finds important addressed politically, scholars of democratic inequalities have almost exclusively focused on representational biases in preference responsiveness, thereby neglecting the crucial aspect of agenda-setting and issue attention at the early stages of policy-making for representation. In this study, I have linked the issues deemed most important by both men and women in the Norwegian public to parliamentary issue attention to explore potential gender biases in priority responsiveness. Despite the democratic ideals of responsiveness and equality, this study suggests a significant gender bias in priority responsiveness, favouring men's issue priorities. 

In Norway's comparatively gender-egalitarian society, I find significant gender differences in issue priorities among Norwegian citizens, with men and women showing distinct priorities across various policy areas. The stability of these gender differences over time suggests that these are not short trends influenced by short-term events. The differences in issue priorities seem related to aspects of Norwegian society where gender roles are more pronounced. Women tend to prioritise issues related to care, family, the public sector, and matters of equality. In contrast, men are more inclined to focus on, among other issues, trade, finance, and infrastructure. These priority gaps may stem from the gender-segregated nature of the workforce and domestic labour. However, further research should be conducted to understand the underlying reasons for why men and women display such differing political issue priorities.

The main question of this paper concerned the extent to which gender biases in priority responsiveness exists. Specifically, whether representatives equally well address men and women's priorities during parliamentary debates. The analyses show that they do not. The proportions of specific topics addressed in the Norwegian \textit{Storting} favours the issue priorities of male citizens over women's issue priorities. Interestingly, this seems to be explained by female representatives being responsive to the public in general, whereas male representatives predominantly cater to the priorities of male citizens. To some extent, this implies that female representatives are better at representing women citizens than their male colleagues. However, this is not because they focus on issues of interest to women over other issues but, compared to their male colleagues, they pay equal attention to the concerns of both genders. 
My findings show that we should be careful making conclusions about the the quality of equal representation by looking only at the extent to which policies reflect the directional preferences of citizens. Even if they do, we need to know how policies align with citizens' preferences on high- or low-priority issues to adequately evaluate representation. That being said, priority responsiveness is not sufficient by itself, for representation is at its best when citizens' preferences are reflected on matters they are truly concerned about. Further, representation is only equal when there are no systematic biases in either priority or policy responsiveness. 

Research on unequal representation would benefit from integrating studies of priority responsiveness with other measures of political representation such as policy responsiveness. This paper presents a template for how we can include a focus on issue priorities. It shows that by combining citizens' inductively measured priorities with large-scale text data from parliamentary debates, we can reveal important findings concerning the state of democratic representation. 

Future research could further explore the role of parties in representing various groups during debates and how party affiliation interacts with gender. Furthermore, while this study has looked at priority responsiveness across various broad topics, our knowledge about gender biases in representation could further benefit from using the same approach to study variation in issue priorities across a smaller subset of issue categories before investigating the extent to which these are reflected in parliamentary debates. What is more, taking a closer look at how gender interacts with other identities on prioritising more specific issues would give us better information about the potential reasons for why gender differences in issue priorities at the citizen level exist, and about the nuances of unequal priority responsiveness. 