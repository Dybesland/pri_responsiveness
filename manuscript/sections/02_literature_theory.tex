% Slapin and Immergut: What are the mechanism? 
% Transmission of information, Party incentives osv..
% See https://www.cambridge.org/core/journals/british-journal-of-political-science/article/which-information-do-politicians-pay-attention-to-evidence-from-a-field-experiment-and-interviews/E3723DD3608DA0BDD74F59F64FF3A45C


\section*{Who is Representing What, When?}
\subsection*{Understanding Women's Substantive Representation}
% \doublespacing
Scholarship on gender and political representation has long been engaged with exploring the conditions under which "women's interests" are represented in the political system, typically referred to as women's substantive representation \parencite{pitkin_concept_1967}. 
As the share of women in national parliaments rises, there is growing interest in whether this increase effectively translates into better representation of women's interests — whether there is a link between descriptive and substantive representation. Scholars anticipate a positive relationship between these two types of representation, primarily based on Phillip's \parencite*{phillips_politics_1998} notion that, due to shared experiences, female representatives are better equipped than their male colleagues to understand and advocate for women's interests \parencite{bratton_descriptive_2002, bolzendahl_womens_2007, brooks_social_2006, clayton_quota_2018}. 

The hypothesised association between descriptive and substantive representation has gained substantial empirical support \parencite[see][for an overview]{wangnerud_women_2009}. For instance, higher numbers of female representatives seem to be associated with increased daycare coverage \parencite{bratton_descriptive_2002} and higher welfare spending \parencite{bolzendahl_womens_2007, brooks_social_2006, clayton_quota_2018}. Additionally, female representatives are found to more frequently engage with issues such as women's equality, children, and family matters compared to their male colleagues, arguably doing more to substantively represent their female constituents \parencite{clayton_quotas_2017,unal_role_2021,schwindt-bayer_still_2006}. Furthermore, elite surveys reveal that female representatives tend to see it as their responsibility to advocate for women's interests \parencite{reingold_concepts_1992}.

Nevertheless, scholars highlight that the mere presence of women in legislative bodies is typically seen as a necessary, but not sufficient, condition for advancing women's interests in politics \parencite{lovenduski_feminizing_2002,clayton_quotas_2017,franceschet_gender_2008}. Beyond women's descriptive representation, factors within the broader institutional context significantly influence women's substantive representation. As \textcite[208]{lovenduski_feminizing_2002} points out, "parliaments are masculine," reflecting the norms of their male founders and historical male dominance. The entrenched norms in many political institutions often clash with societal gender norms about appropriate behaviour for women, thus posing additional challenges for women to fulfil their representative roles \parencite{lovenduski_feminizing_2002, vallejo_vera_politics_2022, franceschet_gender_2008}. Additionally, scholars argue that institutional norms tend to marginalise issues primarily concerning women, viewing them as less prestigious compared to other policy areas and thereby disincentivising both male and female representatives from engaging with such issues \parencite{whip_representing_1991, escobar-lemmon_women_2005,franceschet_themes_2012}.

This scholarship has advanced our understanding of the conditions under which women might benefit from better substantive representation, emphasising the significance of both representative characteristics and institutional dynamics in determining which issues receive attention \parencite[e.g.,][]{clayton_quota_2018} as well a policy output \parencite[e.g.,][]{bratton_descriptive_2002, bolzendahl_womens_2007}. However, work on women's substantive representation is increasingly criticised for the approach to defining "women's interests" or "female-friendly policies".

As pointed out by \textcite[153]{celis_constituting_2014}, women's substantive representation is typically evaluated based on representatives efforts to promote theoretically assumed interests that fall along three lines: women's traditional roles within patriarchal societies, women's participation in the labour market, and women's opportunities to transform their roles to attain greater gender equality. 
Among others, \textcite{celis_constituting_2014, yildirim_rethinking_2022} highlight that by relying on such theoretical assumptions \parencite[e.g.,][]{bratton_descriptive_2002, bolzendahl_womens_2007, brooks_social_2006} or feminist claims \parencite[e.g.,][]{clayton_quotas_2017} about what constitutes women's interests, scholars risks essentialising women and overlooking how interests may vary depending on the context in which they arise. 

Moreover, empirical assessments of gender differences in opinions through surveys have yielded results only partly in compliance with theories about gendered group interests. Gender differences in policy preferences and priorities are found to vary over time and to exist across a range of policy areas, not only in policy domains theorised to be of particular interest to women \parencite{yildirim_rethinking_2022, norris_gender_2003, rosset_how_2019, espirito-santo_gender_2022, gottlieb_men_2018}. This periodic and cross-sectional variation affirms the need to examine gendered opinions inductively before evaluating the extent to which women are substantively represented. 

Furthermore, to evaluate biases in democratic representation according to Dahl's definition of democracy \parencite*[2]{dahl_polyarchy_1971} where unbiased 'responsiveness' to citizens' wishes is essential, we must understand what these wishes are. According to this definition, democratic responsiveness cannot be evaluated solely on the promotion of theoretically assumed interests. 

\subsection*{Evaluating Gender Gaps in Responsiveness}
Intersecting with the literature discussed above on gender and political representation, a growing body of scholarship evaluates representation not by assessing the extent to which representatives advocate for pre-defined interests but by examining how men and women's empirically measured opinions are reflected in policy-making or policy. Within this field, representation is conceptualised as either congruence \parencite{homola_are_2019,dingler_parliaments_2019} or responsiveness \parencite{reher_gender_2018, persson_mans_2023, mathisen_influence_2024}.

Congruence and responsiveness describe different aspects of representation, as outlined by \textcite[343]{peters_democratic_2018}. When there is preference congruence, it means that representatives' preferences match those of the citizens they represent. On the other hand, policy congruence occurs when there is a match between citizens' preferences and the policies or policy outputs enacted by representatives. Preference responsiveness occurs when representatives react to changes in the preferences or views of citizens, for instance, by adapting their positions or policies to reflect shifts in citizens' preferences. Policy responsiveness, similarly, involves policies or policy outputs being adjusted in response to changes in citizen preferences. Hence, the main difference between congruence and responsiveness is whether the relationship between citizens' opinions and opinions or policy decisions at the elite-level is operationalised as static, a mere alignment of citizens' and elites' opinions or policy, or dynamic, where changes at the citizen-level is followed by a change of position at the elite-level or in policy. 

Cross-country studies focusing on gender biases in preference congruence or responsiveness yield mixed results. When measured as the alignment of citizens and parties on a left-right ideological axis, \textcite{bernauer_mind_2015} find no significant congruence gaps, whereas \textcite{ferland_gender_2020} suggests that parties follow ideological shifts of both women and men but are more responsive to men's shifts in ideological position. When looking at citizens' and parties' positions in specific policy domains, \parencite{dingler_parliaments_2019} suggest that parties' positions align slightly better with women's in five of seven policy areas.  

Shifting the focus from preferences to policies, \textcite{reher_gender_2018} investigates policy congruence, measured as the relationship between men and women's support for specific policies and whether such policies are in place across various policy areas in 31 European countries. Her findings suggest that men and women tend to have similar preferences, but when they diverge men's preferences are more often represented. Similarly, \textcite{persson_mans_2023} study policy congruence and responsiveness across 43 countries and find that policies generally align better with men's preferences. Moreover, they observe that the bias is even more pronounced when focusing on whether corresponding policy changes follow preferences.

Narrowing the focus to specific cases, \textcite{mathisen_influence_2024} investigates whether public policy responds equally to the preferences of women and men. He identifies significant gaps in policy responsiveness favouring men in both the United States and Norway. However, in Norway, these gender gaps have substantially decreased over time, in tandem with a higher share of female representatives. Interestingly, among others investigating the impact of women's descriptive representation on observed gender gaps, some find that increased female presence in parliaments is associated with a smaller representational gap \parencite{bernauer_mind_2015, ferland_gender_2020}, whereas others find that it is not \parencite{reher_gender_2018, dingler_parliaments_2019, homola_are_2019}.

Taken together, these findings suggest that there is significant potential for advancing our understanding of the degree to which gender-based disparities in representation exist, under what conditions, and when during the policy-making process. In particular, the stage of the policy process at which representation gaps are explored seems to impact the observed gender gap. Larger gaps are often found concerning final policy output, compared to when gaps are measured as preference congruence or preference responsiveness, underscoring the importance of paying attention to biases at different stages of the policy-making process.  

So far, scholars of unequal responsiveness have mostly overlooked a crucial aspect of policy-making that occurs even before policymakers take a positional stance or enact policies. This initial and crucial step involves agenda-setting. As \textcite{jones_representation_2004} asserts, agenda-setting constitutes an important step in the multifaceted policy-making process as it sets the stage for subsequent policy action because there is little chance of policy change without attention to an issue. Hence, as pointed out by \parencite[278]{jones_representation_2009}, we need to ask: "If representation is limited, is it because issues are denied access to the agenda or because they fail during decision-making?

Due to factors such as institutional rules, time constraints, incentives of vote maximisation, and party competition \parencite{hobolt_government_2008, abou-chadi_brahmin_2021, slapin_introduction_2014}, as well as potential information biases about constituents' concerns \parencite{traber_social_2022}, the capacity of politicians to address all issues is limited and they have to choose to prioritise certain issues over others. This trade-off can result in biases, where legislators pay greater attention to the priorities of certain citizens over others' \parencite{jones_agenda_2018}. Such bias in priority responsiveness, referring to the transmission of the priorities of the public onto the policy priorities of the government, is an essential aspect of democratic (in)equality. For one, attention to political issues constitutes a necessary (though not sufficient) condition for policy change \parencite{jones_representation_2004}, Furthermore, even if preference or policy responsiveness is seemingly equal, there may be severe biases in whether the political issues that are considered in the first place match different groups' views of what constitute high or low priority issues \parencite[2]{jones_representation_2004}. Therefore, an adequate measure of representation includes not only whether public preferences are reflected in policies but also whether they align with the public's prioritised interests.

Generally speaking, public issue priorities tend to impact parties and elected representatives issue agenda \parencite{dennison_explaining_2023, reher_role_2014, kluver_setting_2016,kluver_who_2016, barbera_who_2019}.
However, we know less about how distinct groups' priorities impact politicians' issue priorities differently. So far, there are only a few exceptional studies on biased priority responsiveness, with none focusing on gender. \textcite{traber_social_2022} study how legislative bill introduction reflects the policy priorities of different occupational groups and find that governments pay more attention to what high-status citizens consider important in their legislative agenda but pay less attention to the issues of low-status citizens. Similarly, \textcite{flavin_governments_2017} find that legislators are less likely to act on an issue when low-income citizens prioritise it compared to when affluent citizens do. 

Building upon this limited body of research investigating differential priority responsiveness, I explore gender differential priority responsiveness in Norway, focusing on the extent to which representatives priorities, in terms of issue attention during parliamentary debates, respond equally well to the issue priorities of men and women citizens.

\subsection*{Gender Gaps in Priority Responsiveness - Issue Attention in Norwegian Parliamentary Debates}

Parliamentary debates play a crucial role in political representation, offering a platform for legislators to voice their constituents' concerns, even when they are not part of the ruling government with direct policy influence \parencite{slapin_introduction_2014}. While institutional rules and party leadership may influence the agenda \parencite{slapin_modeling_2014}, parliamentary debates function as an essential space for both parties and individual representatives to draw attention to and increase the salience of, specific issues in the political debate \parencite{ivanusch_issue_nodate, debus_manifestation_2021}. 

In the Norwegian \textit{Storting}, the degree to which the debate agenda is controlled by parties and formal restrictions depends on the type of debate contribution \parencite{soyland_norway_2021}. While parties generally have control over their representative's speech behaviour, individual representatives are given some freedom to emphasise issues, especially during question sessions as these are means of executive control and oversight and functions as important opportunities for setting issues on the agenda \parencite{soyland_party_2019, rasch_behavioural_2011}.\footnote{See \parencite{soyland_norway_2021} for a detailed description of \textit{Stortinget's} institutional setting for parliamentary debates} Furthermore, there are strict behavioural rules during parliamentary debates. All speeches must be addressed to the parliamentary president, and the length of the debate contribution is strictly regulated. The tone must be formal, and other representatives cannot call out or disrupt spontaneously \parencite{soyland_party_2019}.

While party control over the legislative agenda influences the issues that receive attention during parliamentary debates, this should not automatically translate into a representational bias. Parties, with their representatives, are tasked with representing voters, and election turnout is approximately equal between men and women in Norway (slightly more women turn out to vote) \parencite{noauthor_ssbno_2021}. However, parties determine who takes the parliamentary floor during legislative debates. 

Corroborating with existing studies on gender dynamics in parliamentary debates that consistently show that female representatives speak less frequently than their male counterparts during debates \parencite{back_when_2019, back_who_2014, soyland_norway_2021}, \cref{fig:speech_seat} demonstrates that this is also true in Norway. Women deliver fewer speeches than their proportion of seats would suggest. Moreover, while women hold a high share of parliamentary seats compared to most countries, they are still in the minority. Hence, if women who deliver a lower amount of speeches than men prioritise the issues female citizens regard important and men prioritise the issues that men find important, it would result in a priority responsiveness bias favouring men. 

% Speech figure should go around here: 

Furthermore, previous studies quite unanimously find that women are more likely to deliver speeches on subjects such as welfare, health, and family policy. Conversely, men are more inclined to address issues related to finance, energy, and technology \parencite{back_when_2019, unal_role_2021, catalano_women_2009,hargrave_earning_nodate}. Such tendencies have been found in Sweden \parencite{back_politics_2021}, which arguably, similar to Norway, could be considered as a least likely case for finding gender differences in issue attention due to their historically comparative high levels of gender equality \parencite{undp_gender_2024} and high share of female legislators \parencite{ipu_inter_2024}. 


% Perhaps update hypotheses. 
A history of high female presence in the parliament could influence legislative norms and practices to become more favourable to women's issues. Theories of gradual institutional change suggest that marginalised groups can induce changes in institutional norms when they organise and assert their power \parencite{mahoney_theory_2009, dahlerup_small_1988}. This could lead to a legislative environment where women speak more frequently, and both male and female representatives address a broader spectrum of issues.

However, empirical observation of representational biases favouring men \parencite[e.g][]{mathisen_influence_2024, persson_mans_2023, reher_gender_2018}, as well as gender differences in representatives' parliamentary speech behaviour \parencite[e.g.,][]{back_politics_2021, unal_role_2021}, suggest that the political system might be more likely to cater to male citizens opinions over women, even in comparatively gender-egalitarian countries. Hence, I produce the following hypothesis concerning an overall priority responsiveness gap towards men and women citizens in Norway. 

\vspace{5pt}
$H1{a}$: \textit{Representatives show greater priority responsiveness towards male citizens, compared to women, in their parliamentary speeches}
\vspace{5pt}

Furthermore, building on literature exploring the conditions under which women gain better substantive representation \parencite[e.g.,][]{bratton_descriptive_2002, lovenduski_feminizing_2002, clayton_quota_2018}, as well as previous research on gender differences in parliamentarians issue attention during speeches \parencite[e.g.,][]{back_when_2019, unal_role_2021, catalano_women_2009,hargrave_earning_nodate} I expect that disparity in priority responsiveness between male and female citizens can be attributed to representatives predominantly prioritising issues deemed important by constituents whit whom they share the same gender. Hence, the following hypothesis is produced.  

\vspace{5pt}
$H1{b}$: \textit{The gap in priority responsiveness can be explained by representatives delivering speeches on issues that their fellow gendered citizens regard as more important.}
\vspace{5pt}






 