% 240 words
Equal responsiveness is a fundamental principle of democracy, yet researchers question whether men and women have equal chances of having their wishes reflected in policy-making. This paper explores gender biases in priority responsiveness — the extent to which representatives' issue priorities follow the priorities of men and women equally well. Drawing on 123,549 survey responses from Norwegian citizens regarding what they consider the most important political issues, combined with a time-series dataset comprising 111,477 parliamentary speeches given in the Norwegian \textit{Storting} over the same ten-year period, I investigate whether representatives are equally responsive to the priorities of male and female citizens in terms of the issues they address during parliamentary debates. Furthermore, whether representatives are more likely to deliver speeches on topics that are most important to their fellow gendered citizens.

The paper shows that there are significant gender gaps in citizens' policy priorities in the relatively gender-equal context of Norway. Contrasting with democratic ideals, gender differences in issue priorities translate into a gap in priority responsiveness between men and women. Representatives are more likely to allocate attention to the concerns of male citizens than female citizens during speeches. This finding, however, seems to be driven by male representatives who are more inclined to deliver speeches discussing issues that male citizens emphasise as most important, compared to the issues women citizens find most important. Female representatives, on the other hand, show no tendencies to favour one genders' priorities over the other. 